\documentclass[11pt,letterpaper]{article}

\usepackage{subfigure}
\usepackage{graphicx}
\usepackage{amsmath}
\usepackage{amssymb}
\usepackage{enumitem}

\setlength{\textwidth}{6.5in}     
\setlength{\oddsidemargin}{0in}  
\setlength{\evensidemargin}{0in}
\setlength{\textheight}{8.5in} 
\setlength{\topmargin}{0in}   
\setlength{\headheight}{14pt} 
\setlength{\headsep}{10pt}   
%\setlength{\footskip}{0in}

%------------------------------------------------
\newcommand{\homework}[2]{
\setcounter{section}{#1}
\section*{ICS621 Homework {\thesection}: {#2} }
{\markboth{#2}{#2}}
}
%------------------------------------------------
 \addtolength{\parskip}{\baselineskip}

\begin{document}
% Enter the Homework number and title as arguments to
% homework
\homework{2}{Heaps}

%\renewcommand{\baselineskip}{12pt}
\noindent Choose one of the following.

\noindent
\textbf{Problem 6-2 from CLRS: Analysis of d-ary heaps.}  
A \textbf{d-ary heap} is like a binary heap, but (with one
possible exception) non-leaf nodes have $d$ children instead
of 2 children.
\begin{enumerate}[label=\alph*),labelindent=0pt]
\item How would you represent a $d$-ary heap in an array?
\item What is the height of a $d$-ary heap of $n$ elements
in terms of $n$ and $d$ ?
\item Give an efficient implementation of
\textsc{Extract-Max} in a $d$-ary max-heap. Analyze its
running time in terms of $d$ and $n$.
\item Give an efficient implementation of \textsc{Insert} in
a $d$-ary max-heap. Analyze its running time in terms of $d$
and $n$.
\item Give an efficient implementation of
\textsc{Increase-Key($A,i,k$)}, which flags and error if
$k<A[i]$, but otherwise sets $A[i]=k$ and then updates the
$d$-ary max-heap structure appropriately. Analyze its
running time in terms of $d$ and $n$.
\end{enumerate}

\noindent
\textbf{Problem 6-3 from CLRS: Young tableaus.}  
An $m\times n$ \textbf{Young tableau} is an $m\times n$
matrix such that the entries of each row are in sorted order
from left to right and the entries of each column are in
sorted order from top to bottom. Some of the entries of a
Young tableau may be $\infty$, which we treat as nonexistent
elements. Thus, a Young tableau can be used to hold $r \leq
mn$ finite numbers.
\begin{enumerate}[label=\alph*),labelindent=0pt]
\item Draw a $4 \times 4$ Young tableau containing the
elements \{9, 16, 3, 2, 4, 8, 5, 14, 12\}.
\item Argue that an $m\times n$ Young tableau $Y$ is empty
if $Y[1,1]=\infty$. Argue that $Y$ is full (contains $mn$
elements) if $Y[m,n] < \infty$.
\item Give an algorithm to implement \textsc{Extract-Min} on
a nonempty $m\times n$ Young tableau that runs in $O(m+n)$
time. Your algorithm should use a recursive subroutine that
solves an $m\times n$ problem by recursively solving either
an $(m-1)\times n$ or $m\times (n-1)$ subproblem.
(\textit{Hint}: Think about \textsc{Max-Heapify}). Define
$T(p)$, where $p=m+n$, to be the maximum running time of
\textsc{Extract-Min} on any $m\times n$ Young tableau. Give
and solve a recurrence for $T(p)$ that yields the $O(m+n)$
time bound.
\item Show how to insert a new element into a nonfull
$m\times n$ Young tableau in $O(m+n)$ time.
\item Using no other sorting method as a subroutine, show
how to use an $n\times n$ Young tableau to sort $n^2$
numbers in $O(n^3)$ time.
\item Give an $O(m+n)$-time algorithm to determine whether a
given number is stored in a given $m\times n$ Young tableau. 
\end{enumerate}
\end{document}
