\documentclass[11pt,letterpaper]{article}

\usepackage{subfigure}
\usepackage{graphicx}
\usepackage{amsmath}
\usepackage{amssymb}
\usepackage{enumitem}

\setlength{\textwidth}{6.5in}     
\setlength{\oddsidemargin}{0in}  
\setlength{\evensidemargin}{0in}
\setlength{\textheight}{8.5in} 
\setlength{\topmargin}{0in}   
\setlength{\headheight}{14pt} 
\setlength{\headsep}{10pt}   
%\setlength{\footskip}{0in}

%------------------------------------------------
\newcommand{\homework}[2]{
\setcounter{section}{#1}
\section*{ICS621 Homework {\thesection}: {#2} }
{\markboth{#2}{#2}}
}
%------------------------------------------------


 \addtolength{\parskip}{\baselineskip}
\begin{document}

% Enter the Homework number and title as arguments to
% homework
\homework{4}{AVL Trees {\it or} Josephus Permutation}

\noindent Choose one of the following.

\noindent
\textbf{Problem 13-3 from CLRS.} 
An \textbf{AVL tree} is a binary search tree that is height
balanced: for each node $x$, the heights of the left
and right subtrees of $x$ differ by at most 1. To implement
an AVL tree, we maintain an extra attribute in each node:
$x.h$ is the height of node $x$. As for any other binary
search tree $T$, we assume that $T.root$ points to the root
node.
\begin{enumerate}[label=\alph*),labelindent=0pt]
%--------------------------------------
\item Prove that an AVL tree with $n$ nodes has height
$O(\lg n)$. (\textit{Hint:} Prove that an AVL tree of height
$h$ has at least $F_h$ nodes, where $F_h$ is the $h$th
Fibonacci number.)
%--------------------------------------
\item To insert into an AVL tree, we first place a node into
the appropriate place in binary search tree order.
Afterward, the tree might no longer be height balanced.
Specifically, the heights of the left and right children of
some node might differ by 2. Describe a procedure
\textsc{Balance($x$)}, which takes a subtree rooted at $x$
whose left and right children are height balanced and have
heights that differ by at most 2, i.e., $|x.right.h -
x.left.h| \leq 2$, and alters the subtree rooted at $x$ to
be height balanced. (\textit{Hint:} Use rotations.)
%--------------------------------------
\item Using part (b), describe a recursive procedure
\textsc{AVL-Insert($x,z$)} that takes a node $x$ within an
AVL tree and a newly created node $z$ (whose key has already
been filled in), and adds $z$ to the subtree rooted at $x$,
maintaining the property that $x$ is the root of an AVL
tree. As in \textsc{Tree-Insert} from Section~12.3, assume
that $z.key$ has already been filled in and that
$z.left=\text{NIL}$ and $z.right=\text{NIL}$; also assume
that $z.h=0$. Thus, to insert the node $z$ into the AVL tree
$T$, we call \textsc{AVL-Insert($T.root, z$)}.
%--------------------------------------
\item Show that \textsc{AVL-Insert}, run on an $n$-node AVL
tree, takes $O(\lg n)$ time and performs $O(1)$ rotations.
\end{enumerate}

\noindent
\textbf{Problem 14-2 from CLRS.} 
We define the \textbf{Josephus problem} as follows. Suppose
that $n$ people form a circle and that we are given a
positive integer $m \leq n$. Beginning with a designated
first person, we proceed around the circle, removing every
$m$th person. After each person is removed, counting
continues around the circle that remains. This process
continues until we have removed all $n$ people. The order in
which the people are removed from the circle defines the
\textbf{$(n,m)$-Josephus permutation} of the integers
$1,2,\ldots,n$. For example, the $(7,3)$-Josephus
permutation is $\langle3,6,2,7,5,1,4\rangle$. 
\begin{enumerate}[label=\alph*),labelindent=0pt]
\item Suppose that $m$ is a constant. Describe an
$O(n)$-time algorithm that, given integer $n$, outputs the
$(n,m)$-Josephus permutation.
\item Suppose that $m$ is not a constant. Describe an $O(n
\lg n)$-time algorithm that, given integers $n$ and $m$,
outpus the 
$(n,m)$-Josephus permutation.
\end{enumerate}

\end{document}
