\documentclass[11pt,letterpaper]{article}

\usepackage{subfigure}
\usepackage{graphicx}
\usepackage{amsmath}
\usepackage{amssymb}
\usepackage{enumitem}

\setlength{\textwidth}{6.5in}     
\setlength{\oddsidemargin}{0in}  
\setlength{\evensidemargin}{0in}
\setlength{\textheight}{8.5in} 
\setlength{\topmargin}{0in}   
\setlength{\headheight}{14pt} 
\setlength{\headsep}{10pt}   
%\setlength{\footskip}{0in}

%------------------------------------------------
\newcommand{\homework}[2]{
\setcounter{section}{#1}
\section*{ICS621 Homework {\thesection}: {#2} }
{\markboth{#2}{#2}}
}
%------------------------------------------------


\begin{document}

% Enter the Homework number and title as arguments to
% homework
\homework{1}{Insertion sort on small arrays in merge sort}

\textbf{Problem 2-1 from CLRS.} Although merge sort runs in
$\Theta(n \log n)$ worst-case time and insertion sort runs
in $\Theta(n^2)$ worst-case time, the constant factors in
insertion sort can make it faster in practice for small
problem sizes on many machines. Thus, it makes sense to
coarsen the leaves of the recursion by using insertion sort
within merge sort when subproblems become sufficiently
small. Consider a modification to merge sort in which $n/k$
sublists of length $k$ are sorted using insertion sort and
then merged using the standard merging mechanism, where $k$
is a value to be determined.
\begin{enumerate}[label=\alph*),labelindent=0pt]
\item Show that insertion sort can sort the $n/k$ sublists,
each of length $k$, in $\Theta(nk)$ worst-case time.
\item Show how to merge the sublists in $\Theta(n
\log(n/k))$ worst-case time.
\item Given that the modified algorithm runs in
$\Theta(nk+n\log(n/k))$ worst-case time, what is the largest
value of $k$ as a function of $n$ for which the modified
algorithm has the same running time as standard merge sort,
in terms of $\Theta$-notation?
\item How should we choose $k$ in practice?
\end{enumerate}

\end{document}
