\documentclass[11pt,letterpaper]{article}

\usepackage{subfigure}
\usepackage{graphicx}
\usepackage{amsmath}
\usepackage{amssymb}
\usepackage{enumitem}

\setlength{\textwidth}{6.5in}     
\setlength{\oddsidemargin}{0in}  
\setlength{\evensidemargin}{0in}
\setlength{\textheight}{8.8in} 
\setlength{\topmargin}{0in}   
\setlength{\headheight}{14pt} 
\setlength{\headsep}{10pt}   
%\setlength{\footskip}{0in}

%------------------------------------------------
\newcommand{\homework}[2]{
\setcounter{section}{#1}
\section*{ICS621 Homework {\thesection}: {#2} }
{\markboth{#2}{#2}}
}
%------------------------------------------------


\addtolength{\parskip}{\baselineskip}
\begin{document}

% Enter the Homework number and title as arguments to
% homework
\homework{6}{Dynamic Programming}

\noindent
Choose one of the following.

\noindent
\textbf{Problem 15-6 from CLRS. Planning a company party.} 
Professor Stewart is consulting for the president of a
corporation that is planning a company party. The company
has a hierarchical structure; that is, the supervisor
relation forms a tree rooted at the president. The personnel
office has ranked each employee with a conviviality rating,
which is a real number. In order to make the party fun for
all attendees, the president does not want both an employee
and his or her immediate supervisor to attend.

Professor Stewart is given the tree that describes the
structure of the corporation, using the left-child,
right-sibling representation described in Section 10.4. Each
node of the tree holds, in addtion to the pointers, the name
of an employee and that employee's conviviality ranking.
Describe an algorithm to make up a guest list that maximizes
the sum of the conviviality ratings of the guests. Analyze
the running time of your algorithm. 

\noindent
\textbf{Problem 15-10 from CLRS. Planning an investment
strategy.} 
Your knowledge of algorithms helps you obtain an exciting
job with the Acme Computer Company, along with a \$10,000
signing bonus. You decide to invest this money with the goal
of maximizing your return at the end of 10 years. You decide
to use the Amalgamated Investment Company to manage your
investments. Amalgamated Investments requires you to observe
the following rules. It offers $n$ different investments,
numbered $1$ through $n$. In each year $j$, investment $i$
provides a return rate $r_{ij}$. In other words, if you
invest $d$ dollars in investment $i$ in year $j$, then at
the end of year $j$, you have $dr_{ij}$ dollars. The return
rates are guaranteed, that is, you are given all the return
rates for the next 10 years for each investment. You can
make investment decisions only once per year. At the end of
each year, you can leave the money made in the previous year
in the same investments, or you can shift the money to other
investments, by either shifting money between existing
investments or moving money to a new investment. If you do
not move your money betweem two consecutive years, you pay a
fee of $f_1$ dollars, whereas if you switch your money, you
pay a fee of $f_2$ dollars, where $f_2>f_1$.
\begin{enumerate}[label=\alph*),labelindent=0pt, itemsep=0pt]
%--------------------------------------
\item The problem, as stated, allows you to invest your
money in multiple investments in each year. Prove that there
exists an optimal investment strategy that, in each year,
puts all the money into a single investment. (Recall that an
optimal investment strategy maximizes the amount of money
after 10 years and is not concerned with any other
objectives, such as minimizing risk.)
\item Prove that the problem of planning your optimal
investment strategy exhibits optimal substructure.
\item Design an algorithm that plans your optimal investment
strategy. What is the running time of your algorithm?
\item Suppose that Amalgamated Investments imposed the
additional restriction that, at any point, you can have no
more than \$15,000 in any one investment. Show that the
problem of maximizing your income at the end of 10 years no
longer exhibits optimal substructure.
%--------------------------------------
\end{enumerate}
\end{document}
