\documentclass[11pt,letterpaper]{article}

\usepackage{subfigure}
\usepackage{graphicx}
\usepackage{amsmath}
\usepackage{amssymb}
\usepackage{enumitem}

\setlength{\textwidth}{6.5in}     
\setlength{\oddsidemargin}{0in}  
\setlength{\evensidemargin}{0in}
\setlength{\textheight}{8.5in} 
\setlength{\topmargin}{0in}   
\setlength{\headheight}{14pt} 
\setlength{\headsep}{10pt}   
%\setlength{\footskip}{0in}

%------------------------------------------------
\newcommand{\homework}[2]{
\setcounter{section}{#1}
\section*{ICS621 Homework {\thesection}: {#2} }
{\markboth{#2}{#2}}
}
%------------------------------------------------


\begin{document}

% Enter the Homework number and title as arguments to
% homework
\homework{1}{Merge sort \& Inversions}

\textbf{Problem 2-1 from CLRS.} Although merge sort runs in
$\Theta(n \log n)$ worst-case time and insertion sort runs
in $\Theta(n^2)$ worst-case time, the constant factors in
insertion sort can make it faster in practice for small
problem sizes on many machines. Thus, it makes sense to
coarsen the leaves of the recursion by using insertion sort
within merge sort when subproblems become sufficiently
small. Consider a modification to merge sort in which $n/k$
sublists of length $k$ are sorted using insertion sort and
then merged using the standard merging mechanism, where $k$
is a value to be determined.
\begin{enumerate}[label=\alph*),labelindent=0pt]
\item Show that insertion sort can sort the $n/k$ sublists,
each of length $k$, in $\Theta(nk)$ worst-case time.
\item Show how to merge the sublists in $\Theta(n
\log(n/k))$ worst-case time.
\item Given that the modified algorithm runs in
$\Theta(nk+n\log(n/k))$ worst-case time, what is the largest
value of $k$ as a function of $n$ for which the modified
algorithm has the same running time as standard merge sort,
in terms of $\Theta$-notation?
\item How should we choose $k$ in practice?
\end{enumerate}

\noindent
\textbf{Problem 2-4 from CLRS: Inversions.}
Let $A[1\ldots n]$ be an array of $n$ distinct numbers. If
$i < j$ and $A[i]> A[j]$, then the pair $(i,j)$  is called
an inversion of $A$.
\begin{enumerate}[label=\alph*),labelindent=0pt]
\item List the five inversions of the array $\langle 2, 3,
        8, 6, 1\rangle$.
\item What array with elements from the set $\{1,2, \ldots
        n\}$ has the most inversions?  How many does it
        have?
\item What is the relationship between the running time of
insertion sort and the number of inversions in the input
array? Justify your answer.
\item Give an algorithm that determines the number of
inversions in any permutation
on $n$ elements in $O(n lg n)$ worst-case time. (Hint: Modify
merge sort.)
\end{enumerate}
You only need to write the answer for part (d). Remember you
need to prove both correctness and efficiency.

\end{document}
