\documentclass[11pt,letterpaper]{article}

\usepackage{subfigure}
\usepackage{graphicx}
\usepackage{amsmath}
\usepackage{amssymb}
\usepackage{enumitem}

\setlength{\textwidth}{6.5in}     
\setlength{\oddsidemargin}{0in}  
\setlength{\evensidemargin}{0in}
\setlength{\textheight}{8.5in} 
\setlength{\topmargin}{0in}   
\setlength{\headheight}{14pt} 
\setlength{\headsep}{10pt}   
%\setlength{\footskip}{0in}

%------------------------------------------------
\newcommand{\homework}[2]{
\setcounter{section}{#1}
\section*{ICS621 Homework {\thesection}: {#2} }
{\markboth{#2}{#2}}
}
%------------------------------------------------


 \addtolength{\parskip}{\baselineskip}
\begin{document}

% Enter the Homework number and title as arguments to
% homework
\homework{2}{Water Jugs \& AVL Trees}

\noindent
\textbf{Problem 8-4 from CLRS.} 
Suppose that you are given $n$ red and $n$ blue water jugs,
all of different shapes and sizes. All red jugs hold
different amounts of water, as do the blue ones. Moreover,
for every red jug, there is a blue jug that holds the same
amount of water, and vice versa. Your task is to find a
grouping of the jugs into pairs of red and blue jugs that
hold the same amount of water. To do so, you may perform the
following operation: pick a pair of jugs in which one is
red and one is blue, fill the red jug with water, and then
pour the warer into the blue jug. This operation will tell
you whether the red or the blue jug can hold more water, or
they have the same volume. Assume that such a comparison
takes one time unit. Your goal is to find an algorithm that
makes a minimum number of comparisons to determine the
grouping. Remember that you may not directly compare two red
jugs or two blue jugs.
\begin{enumerate}[label=\alph*),labelindent=0pt]
\item Describe a deterministic algorithm that uses
$\Theta(n^2)$ comparisons to group the jugs into pairs.
\item Prove a lower bound of $\Omega(n \lg n)$ for the
number of comparisons that an algorithm solving this problem
must make.
\item Give a randomized algorithm whose expected number of
comparisons is $O(n \lg n)$, and prove that this bound is
correct. What is the worst-case number of comparisons for
your algorithm?
\end{enumerate}

\noindent
\textbf{Problem 13-3 from CLRS.} 
An \textbf{AVL tree} is a binary search tree that is height
balanced: for each node $x$, the heights of the left
and right subtrees of $x$ differ by at most 1. To implement
an AVL tree, we maintain an extra attribute in each node:
$x.h$ is the height of node $x$. As for any other binary
search tree $T$, we assume that $T.root$ points to the root
node.
\begin{enumerate}[label=\alph*),labelindent=0pt]
%--------------------------------------
\item Prove that an AVL tree with $n$ nodes has height
$O(\lg n)$. (\textit{Hint:} Prove that an AVL tree of height
$h$ has at least $F_h$ nodes, where $F_h$ is the $h$th
Fibonacci number.)
%--------------------------------------
\item To insert into an AVL tree, we first place a node into
the appropriate place in binary search tree order.
Afterward, the tree might no longer be height balanced.
Specifically, the heights of the left and right children of
some node might differ by 2. Describe a procedure
\textsc{Balance($x$)}, which takes a subtree rooted at $x$
whose left and right children are height balanced and have
heights that differ by at most 2, i.e., $|x.right.h -
x.left.h| \leq 2$, and alters the subtree rooted at $x$ to
be height balanced. (\textit{Hint:} Use rotations.)
%--------------------------------------
\item Using part (b), describe a recursive procedure
\textsc{AVL-Insert($x,z$)} that takes a node $x$ within an
AVL tree and a newly created node $z$ (whose key has already
been filled in), and adds $z$ to the subtree rooted at $x$,
maintaining the property that $x$ is the root of an AVL
tree. As in \textsc{Tree-Insert} from Section~12.3, assume
that $z.key$ has already been filled in and that
$z.left=\text{NIL}$ and $z.right=\text{NIL}$; also assume
that $z.h=0$. Thus, to insert the node $z$ into the AVL tree
$T$, we call \textsc{AVL-Insert($T.root, z$)}.
%--------------------------------------
\item Show that \textsc{AVL-Insert}, run on an $n$-node AVL
tree, takes $O(\lg n)$ time and performs $O(1)$ rotations.
\end{enumerate}


\end{document}
